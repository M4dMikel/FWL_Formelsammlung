\section{Antennen}
\subsection{Herz'scher Dipol}
\boxed{\vec{p} = Q\cdot\vec{d}}
\subsubsection{Allgemein}

{\footnotesize\begin{empheq}[box=\fbox]{align*}
        {\vec{H}} & =-\frac{I_0\Delta l'\beta^2}{4\pi}e^{-j\beta R}\cdot\sin\theta\left(\frac{1}{j\beta R}+\frac{1}{(j\beta R)^2}\right)\vec{e}_\phi                                 \\
        {\vec{E}} & = -\frac{Z_F I_0\Delta l'\beta^2}{2\pi}e^{-j\beta R}\cdot\cos\theta\left(\frac{1}{(j\beta R)^2}+\frac{1}{(j\beta R)^3}\right)\vec{e}_R                           \\
        & = -\frac{Z_F I_0\Delta l'\beta^2}{4\pi}e^{-j\beta R}\cdot\sin\theta\left(\frac{1}{(j\beta R)}+\frac{1}{(j\beta R)^2}+\frac{1}{(j\beta R)^3}\right)\vec{e}_\theta
    \end{empheq}}%

\subsubsection[Nahfeld]{Nahfeld(Fresnel-Zone):\\ $\frac{\lambda}{2\pi R}\gg 1$ oder $\beta R \ll 1$}

Überwiegend \textbf{Blindleistungsfeld}, da $E$ zu $H$ $90^\circ$
phasenverschoben
\begin{empheq}[box=\fbox]{align*}
    \vec{H} & \approx \frac{I_0 \Delta l'}{4\pi R^2}\cdot\sin\theta\cdot\vec{e}_\phi                                            \\
    \vec{E} & \approx \frac{I_0 \Delta l'}{2\pi j \omega\varepsilon R^3}\cos\theta\cdot\vec{e}_R\\
    & +       \frac{I_0 \Delta l'}{4\pi j \omega\varepsilon R^3}\sin\theta\cdot\vec{e}_\theta
\end{empheq}

\subsubsection[Fernfeld]{Fernfeld(Fraunhofer-Zone):\\ $\frac{\lambda}{2\pi R}\ll 1$ oder $\beta R\gg 1$}

Überwiegend \textbf{Wirkleistungsfeld}, $\vec{S}$ nach außen somit Kugelwelle

\vspace{1ex}
mit $\eta = Z_{F0}$

\begin{empheq}[box=\fbox] {align*}
    H & \approx  j\frac{\beta I_0 \Delta l'}{4\pi R}\cdot e^{-j\beta R}\cdot\sin\theta\cdot\vec{e}_\phi                           \\
    E & \approx  j\frac{\beta Z_F I_0 \Delta l'}{4\pi R}\cdot e^{-j\beta R}\cdot\sin\theta\cdot \vec{e}_\theta
\end{empheq}

\subsubsection{Abgestrahlte Leistung im Fernfeld}
\begin{align*}
    P_\texttt{rad} & = \frac{Z_{F0} {I_0}^2 \beta^2 (\Delta l')^2}{12\pi}                             \\
                   & = \frac{I_0^2 Z_F\pi}{3}\cdot \dfrac{\Delta l'^2}{\lambda^2}                     \\
                   & = 40\pi^2\Omega\cdot\left(\frac{I_0\Delta l'}{\lambda}\right)^2                  \\
    S_{av}         & = \frac{Z_FI_0^2\beta^2(\Delta l')^2}{32\pi^2R^2}\cdot\sin^2\theta\cdot\vec{e}_R \\
                   & = \frac{1}{2}\Re\left\{\vec{E}\times\vec{H}^*\right\}
\end{align*}
\subsubsection{Strahlungswiderstand}
\begin{align*}
    R_{S} & = \frac{2}{3}\pi Z_F\left(\frac{\Delta l'}{\lambda}\right)^2
    = 80\pi^2\Omega\left(\frac{\Delta l'}{\lambda}\right)^2
\end{align*}
\subsubsection{Verlustwiderstand}
\begin{align*}
    R_{v} & = \frac{l}{\sigma\cdot A_\delta}
\end{align*}
\subsection{Magnetischer Dipol}
\boxed{\vec{m} = \vec{I}\pi\vec{a}^2\vec{e}_z}
\boxed{m = I\cdot A}
\begin{center}
    \usetikzlibrary{shapes.geometric}

\tikzset{elliparc/.style args={#1:#2:#3}{%
insert path={(#1:#3) arc (#1:#2:#3)}}}
\begin{tikzpicture}

    %Elipse
    \node[ellipse,
        draw = black,
        dashed,
        minimum width = 3cm,
        minimum height = 1cm] (e) at (0,0) {};
    
    %Punkte
    \fill[](0,0)        circle(.05);
    \fill[](1.75,2.5)   circle(.05)
        node[right]{\tiny{P(R,$\theta$,$\phi$)}};
    \fill[](1.25,0.255) circle(.05)
        node[above right]{\tiny{P'}};
    \fill[](1.75,0.375) circle(.05)
        node[above right]{\tiny{P''}};

    %Koordinaten Achsen
    \draw[-latex](0,0)--(-1.5,-1) node[right]{x};
    \draw[-latex](0,0)--(1.5,-1) node[left]{y};
    \draw[-latex](0,0)--(0,3) node[left]{z};
    
    %Pfeile
    \draw[-latex, blue](0,0)--(1.75,2.5)
        node[above, midway, left]{\tiny{$\vec{R}$}};
    \draw[-latex, blue](1.25,0.255)--(1.75,2.5)
        node[midway, below left]{\tiny{$\vec{R}_1$}};
    \draw[-latex, blue](0,0) -- (1.25,0.255)
        node[above, midway]{\tiny{\tiny{R}'}};
    \draw[-latex, red](-0.9,0.4)--(-.95,.385)
        node[above]{I};
    \draw[-latex](1.25,0.255)--(1,0.4)
        node[above]{\tiny{dl'}};
    \draw[-latex](1.75,2.5)--(1.5,2.65)
        node[above]{\tiny$\vec{A}$};
    \draw[-latex](0,0)--(-1.5,-0)
        node[above, midway]{\tiny$\vec{a}$};
   
    %Winkel
    \draw[-] (90:1) arc (90:55:1)
    node[above, midway] {\tiny{$\theta$}};
    \draw[] (0,0) [elliparc=-117:30:.75cm and .25cm];
    \node[yshift=-4.5]{\tiny{$\phi$'}};
    
    %Legende
    \node[right] at (-0.75,-0.75){\tiny{$\vec{m}=\vec{I}\pi a^2\vec{e}_z$}};
   
    %Linien
    \draw[dashed](0,0)--(2.5,0.5);
    \draw[dashed](1.75,0.375)--(1.75,2.5);

\end{tikzpicture}

\end{center}

\begin{align*}
    \vec{A}  & = \frac{\mu m}{4\pi R^2}(1+j\beta R) e^{-j\beta R}\sin\theta\cdot\vec{e}_\phi \\
    \Delta l & \rightarrow \beta\pi\ a^2
\end{align*}

{\footnotesize\begin{empheq}[box=\fbox]{align*}
    {\vec{H}}   & = -\frac{j\omega\mu\beta^2m}{2\pi Z_{F0}}e^{-j\beta R}\cdot\cos\theta\left(\frac{1}{(j\beta R)^2}+\frac{1}{(j\beta R)^3}\right)\vec{e}_R                             \\
    & = -\frac{j\omega\mu\beta^2m}{4\pi Z_{F0}}e^{-j\beta R}\cdot\sin\theta\left(\frac{1}{(j\beta R)}+\frac{1}{(j\beta R)^2}+\frac{1}{(j\beta R)^3}\right)\vec{e}_\theta   \\
    {\vec{E}}   & =  \frac{j\omega\mu\beta^2m}{4\pi}e^{-j\beta R}\sin\theta\left(\frac{1}{j\beta R}+\frac{1}{(j\beta R)^2}\right)\vec{e}_\phi
\end{empheq}}%

\subsubsection{Fernfeld}
\begin{empheq}[box=\fbox]{align*}
    E & \approx -\frac{\beta m\omega\mu}{4\pi R}e^{-j\beta R}\sin\theta\cdot\vec{e}_\phi \\
    H & \approx -\frac{\beta m\omega\mu}{4\pi R Z_{F0}}e^{-j\beta R}\sin\theta\cdot\vec{e}_\theta
\end{empheq}
\subsubsection{Abgestrahlte Leistung im Fernfeld}
\begin{align*}
    P_\texttt{rad} & = \frac{Z_F\beta^4m^2}{12\pi}                                     \\
                   & = \frac{m^2\mu\omega^4}{12\pi v_p^3}                              \\
    S_{av}         & = \frac{Z_F\beta^4m^2}{32\pi^2R^2}\cdot\sin^2\theta\cdot\vec{e}_R \\
                   & = \frac{1}{2}\Re\left\{\vec{E}\times\vec{H}^*\right\}
\end{align*}

\subsubsection{Nahfeld}
\begin{empheq}[box=\fbox]{align*}
    E & \approx -\frac{jm\omega\mu}{4\pi R^2}\sin\theta\cdot\vec{e}\phi \\
    H & \approx \frac{m}{4\pi R^3}(2\cos\theta\cdot\vec{e}_R+\sin\theta\cdot\vec{e}_\theta)
\end{empheq}

\subsection{Lineare Antenne}
\begin{align*}
    I(z') & = I_0\cdot\sin\left[\beta\left(\frac{L}{2}-|z'|\right)\right]
\end{align*}

\subsubsection{Dipolantenne}
\begin{align*}
    \vec{H}      & = j\cdot\frac{I_0}{2\pi R}\cdot e^{-j\beta R}\cdot\frac{\cos\left[\left(\frac{\beta L}{2}\right)\cos\theta\right]-\cos\left(\frac{\beta L}{2}\right)}{\sin\theta}\cdot\vec{e}_\phi         \\
    \vec{E}      & = H\cdot Z_F\cdot\vec{e}_\theta                                                                                                                                                            \\
    I_0          & = \sqrt{\frac{2\cdot P_{Send}}{R_S}}                                                                                                                                                       \\
                 & \text{Die mittlere Strahlungsleistungsdichte}                                                                                                                                              \\
    \vec{S}_{av} & = \frac{Z_FI_0^2}{8\pi^2 R^2}\left(\frac{\cos\left(\frac{\beta L}{2}\cos\theta\right)-\cos\left(\frac{\beta L}{2}\right)}{\sin\theta}\right)^2\cdot\vec{e}_R                               \\
    \vec{S}_{av} & = S_{max} \cdot D(\vartheta, \varphi) = S_{max} \cdot C(\vartheta, \varphi)^2 
                 & \text{Die gesamte Strahlungsleistung}                                                                                                                                                      \\
    P_S          & = \frac{Z_{F0}I_0^2}{4\pi}\int^{\theta=\pi}_{\theta=0}\frac{\left(\cos\left(\frac{\beta L}{2}\cos\theta\right)-\cos\left(\frac{\beta L}{2}\right)\right)^2}{\sin\theta}\cdot\vec{e}_\theta \\
                 & = \int_A S_{AV}\cdot d\vec{a}                                                                                                                                                              \\
                 & = \int^{2\pi}_{\Phi = 0}\int^{\pi}_{\Theta = 0} S_{AV} R^2 \sin\Theta \cdot d\Theta \cdot d\Phi
\end{align*}

\subsection{Antennenkenngrößen}

\makebox[0pt][l]{
    \begin{minipage}[b]{0.25\columnwidth}
        \resizebox{1.85\columnwidth}{!}{
            \begin{circuitikz}
                \draw(0,3.5) node[below]{$l$}
                to[R=$R_V$](3,3.5)
                to[R=$R_S$](3,1.5)
                to[L=$jX_A$](3,0)
                to[short](0,0) node[above]{$l'$}
                to[open, o-o](0,3.5);
                \draw[latex-, double](1.5,1.75)--(0.5,1.75) node[left]{$\underline{Z}_A$};
            \end{circuitikz}
        }
    \end{minipage}
    \begin{minipage}[b]{0.25\columnwidth}
        \begin{tikzpicture}
            \node[]{};
        \end{tikzpicture}
    \end{minipage}

    \begin{minipage}[b]{0.5\columnwidth}
        \footnotesize
        \begin{align*}
            \underline{Z}_A & := \text{Antennenimpedanz}        \\
            R_V             & := \text{Verlustwiderstand}       \\
            R_S             & := \text{Strahlungswiderstand}    \\
            X_A             & := \text{Antennenblindwiderstand} \\
            D               & := \text{Directifity/Richtfaktor} \\
            G               & := \text{Gain/Gewinn}             \\
            A_{eff}         & := \text{Wirksame Antennenfläche} \\
        \end{align*}
        \normalsize
    \end{minipage}
}

\subsubsection{Abgestrahlte Leistung}
\begin{align*}
    P_S & = \frac{1}{2}\cdot I_A^2 \cdot R_S
\end{align*}

\subsubsection{Verlustleistung}
\begin{align*}
    P_V & = \frac{1}{2}\cdot I_A^2\cdot R_V
\end{align*}

\subsubsection{Wirkungsgrad}
\begin{align*}
    \eta & = \frac{P_S}{P_S + P_V} = \frac{R_S}{R_S + R_V}
\end{align*}

\subsubsection{Richtcharakteristik}
$C_{i} \ent$ isotroper Kugelstrahler als Bezugsgröße in Hauptabstrahlrichtung
\begin{align*}
    C(\vartheta, \varphi)     & = \frac{E(\vartheta, \varphi)}{E_{\max}}=\frac{H(\vartheta, \varphi)}{H_{\max}} = \frac{U(\varphi,\vartheta)}{U_{\max}} & 0 \leq C(\vartheta, \varphi) \leq 1 \\
    C_{i}(\vartheta, \varphi) & = \frac{E(\vartheta, \varphi)}{E_{i}}=\frac{H(\vartheta, \varphi)}{H_{i}}                                               & C_{i}>1
\end{align*}

\subsubsection{Richtfunktion/Richtfaktor}
In $[\si{dB}]$ angeben!
\begin{align*}
    D(\vartheta, \varphi) & = \frac{S(\vartheta, \varphi)}{S_{i}}                               \\
    D(\vartheta, \varphi) & = C_{i}^{2}(\vartheta, \varphi) = D \cdot C^{2}(\vartheta, \varphi) \\
    D                     & = \max \{D(\vartheta, \varphi)\} = \frac{S_{\max}}{S_{i}}
\end{align*}

\subsubsection{Gewinn}
\begin{align*}
    G & = \eta \cdot D \qquad [\si{dB}] \\
\end{align*}

\subsubsection{Wirksame Antennenfläche}
\begin{align*}
    A_\texttt{eff} & = \frac{\lambda^2}{4\pi}\cdot G = \dfrac{Z_{F0}}{4 R_S} \cdot l_\texttt{eff}^2
\end{align*}

\subsection{Bezugsantennen}
\[
    \boxed{g = 10 \cdot log(G) \si{dB}}
\]

mit $P_0$ : Eingangsleistung der Antenne

\begin{description}
    \item \textbf{\underline{G$\rightarrow$Bezugsantenne:}}

          Elementardipol  zu Kugelstrahler \[D = 1,50 \rightarrow g = 1,76\si{dBi}\]
          Halbwellendipol zu Kugelstrahler \[D = 1,64 \rightarrow g = 2,15\si{dBi}\]

    \item \textbf{\underline{EIRP}: Eqivalent \underline{Isoropic} Radiated Power}
          \[
              \text{EIRP} = P_0 \cdot G_i [\si{dBi}]
          \]

    \item \textbf{\underline{ERP}: Eqivalent Radiated Power (verlustloser Halbwellendipol)}
          \[
              \text{ERP} = P_0 \cdot G_d [\si{dBd}]
          \]
\end{description}

\subsection{Senden und Empfangen}
\begin{center}
    \resizebox{0.75\columnwidth}{!}{
        \begin{circuitikz}
            %Schaltbild
            \draw(0,0) node[below right]{$U_0=l_{\texttt{eff}}\cdot E_0$}
            to[V,v=$U_0 $](0,2)                               %Spannungsquelle
            to[R=$R_{s}$](2,2)                                   %Strahlungswiderstand
            to[short](4 ,2)
            to[R= $R_{s}$](4,0)
            to[short](0,0);
            \draw(3,2) node[below]{$1$}
            to[open,o-o](3,0) node[above]{$1'$};

        \end{circuitikz}
    }
\end{center}


\begin{description}
    \item Senden = transmit = TX
    \item Empfangen = receive = RX
\end{description}

\begin{align*}
    \frac{P_{RX}}{P_{TX}}          & = A_{\texttt{eff},RX}\cdot A_{\texttt{eff},TX}\cdot\frac{1}{\lambda^2r^2}                          \\
                                   & = D_{i,RX}\cdot\eta_{RX}\cdot D_{i,TX}\cdot\eta_{TX}\cdot\left(\frac{\lambda}{4\pi r}\right)^2     \\
    \Aboxed{A_\texttt{eff}(\theta) & = G_{RX}\cdot\frac{\lambda^2}{4\pi}\overbrace{\cdot\frac{3}{2}\cdot\sin^2\theta}^{{D_{i,\theta}}}} \\
    P_{RX}                         & = S_{RX}\cdot A_\texttt{eff}                                                                       \\
                                   & = P_{TX}\cdot G_{TX}\cdot G_{RX}\cdot \left(\frac{\lambda}{4\pi r}\right)^2
\end{align*}

\subsubsection{Freiraumdämpfung/Freiraumdämpfungsmaß}
\begin{align*}
    F = \dfrac{P_{TX}}{P_{RX}} \cdot \left(\dfrac{4 \pi d}{\lambda}\right)^2                         & \qquad [1]       \\
    a_{0} = 20 \lg \left(\frac{4 \pi d}{\lambda}\right) =20 \lg \left(\frac{4 \pi d f}{c_{0}}\right) & \qquad [\si{dB}]
\end{align*}

\subsubsection{Leistungspegel/Freiraumpegel}
\begin{align*}
    L      & = 10 \lg \left(\frac{P}{1 \si{mW}}\right) \qquad [\si{dBm}] \\
    L_{RX} & = L_{TX}+g_{TX}+g_{RX}-a_{0} \qquad [\si{dB}]
\end{align*}
