\section{Antennen}

\subsection{Antennenkenngrößen}

\makebox[0pt][l]{
    \begin{minipage}[b]{0.25\columnwidth}
        \resizebox{1.85\columnwidth}{!}{
            \begin{circuitikz}
                \draw(0,3.5) node[below]{$l$}
                to[R=$R_V$](3,3.5)
                to[R=$R_S$](3,1.5)
                to[L=$jX_A$](3,0)
                to[short](0,0) node[above]{$l'$}
                to[open, o-o](0,3.5);
                \draw[latex-, double](1.5,1.75)--(0.5,1.75) node[left]{$\underline{Z}_A$};
            \end{circuitikz}
        }
    \end{minipage}
    \begin{minipage}[b]{0.25\columnwidth}
        \begin{tikzpicture}
            \node[]{};
        \end{tikzpicture}
    \end{minipage}

    \begin{minipage}[b]{0.5\columnwidth}
        \footnotesize
        \begin{align*}
            \underline{Z}_A & := \text{Antennenimpedanz}        \\
            R_V             & := \text{Verlustwiderstand}       \\
            R_S             & := \text{Strahlungswiderstand}    \\
            X_A             & := \text{Antennenblindwiderstand} \\
            D               & := \text{Directifity/Richtfaktor} \\
            G               & := \text{Gain/Gewinn}             \\
            A_{eff}         & := \text{Wirksame Antennenfläche} \\
        \end{align*}
        \normalsize
    \end{minipage}
}

\subsubsection{Abgestrahlte Leistung}
\begin{align*}
    P_S & = \frac{1}{2}\cdot I_A^2 \cdot R_S
\end{align*}

\subsubsection{Verlustleistung}
\begin{align*}
    P_V & = \frac{1}{2}\cdot I_A^2\cdot R_V
\end{align*}

\subsubsection{Wirkungsgrad}
\begin{align*}
    \eta & = \frac{P_S}{P_S + P_V} = \frac{R_S}{R_S + R_V}
\end{align*}

\subsubsection{Richtfaktor und Gewinn}
\begin{align*}
     & D\left[\si{dB}\right]  &  & G\left[\si{dB}\right]  \\
     & d\left[\si{dBi}\right] &  & g\left[\si{dBi}\right] \\
     &                        &  & G  = \eta\cdot D       \\
     & d = 10\cdot\log_{10} D &  & g  = 10\cdot\log_{10}G
\end{align*}

\subsubsection{Wirksame Antennenfläche}
\begin{align*}
    A_W & = \frac{\lambda^2}{4\pi}\cdot G = \dfrac{Z_{F0}}{4 R_S} \cdot l_{eff}^2
\end{align*}

\subsection{Herz'scher Dipol}

\subsubsection{Allgemein}

\begin{empheq}[box=\fbox]{align*}
    &\underline{E}_{r}=\frac{Z_{F} \underline{I} \ell}{2 \pi r^{3} k} \cos \theta \cdot e^{-j k r} [-\mathrm{j}+k r]\\
    &\underline{E}_{\theta}=\frac{Z_{F} \underline{I} \ell}{4 \pi r^{3} k} \sin \theta \cdot e^{-j k r} \left[k r-\mathrm{j}+\mathrm{j}(k r)^{2}\right]\\
    &\underline{H}_{\varphi}=\frac{\underline{I} \ell}{4 \pi r^{2}} \sin \theta \cdot e^{-j k r} [1+\mathrm{j} k r]
\end{empheq}

\subsubsection[Nahfeld]{Nahfeld: $r \ll \lambda$}

Überwiegend \textbf{Blindleistungsfeld}, da $E$ zu $H$ $90^\circ$
phasenverschoben
\begin{empheq}[box=\fbox]{align*}
    H & = \vec{\Phi}\cdot\frac{I_0 \Delta l}{4\pi R^2}\cdot \sin\theta                                                \\
    E & = \frac{I_0 \Delta l}{4\pi j \omega\varepsilon R^3}(2\vec{R} \cdot \cos\theta + \vec{\theta}\cdot \sin\theta)
\end{empheq}

\begin{empheq}[box=\fbox]{align*}
    E_r       & = \frac{I l \lambda}{4\pi^2\varepsilon_0  c_0}\cdot \frac{1}{r^3} \cdot\cos\theta \cdot \sin(\omega t - \beta r) \\
    E_\theta       & = \frac{I l \lambda}{4\pi^2\varepsilon_0  c_0}\cdot \frac{1}{r^3} \cdot\sin\theta \cdot \sin(\omega t - \beta r) \\
    H_\varphi & = \frac{I l}{4\pi}\cdot \frac{1}{r^2}\cdot\sin\varphi\cdot\cos(\omega t -\beta r)
\end{empheq}

\subsubsection[Fernfeld]{Fernfeld: $r\gg\lambda$}

Überwiegend \textbf{Wirkleistungsfeld}, $\vec{S}$ nach außen somit Kugelwelle

\vspace{1ex}
mit $\eta = Z_{F0}$

\begin{empheq}[box=\fbox] {align*}
    H & = \vec{\Phi}\cdot j\frac{\beta I_0 \Delta l}{4\pi R}\cdot \sin\theta                             \\
    E & = \vec{\theta}\cdot j\frac{\beta I_0 \Delta l }{4\pi R} \cdot \sin\theta \cdot\eta e^{-j\beta R}
\end{empheq}

\begin{empheq}[box=\fbox]{align*}
    E_r       & = 0                                                                                                           \\
    E_\theta       & = -\frac{I l }{2\varepsilon_0  c_0 \lambda}\cdot \frac{1}{r} \cdot\sin\theta \cdot \sin(\omega t - \beta r) \\
    H_\varphi & = -\frac{I l}{2\lambda}\cdot \frac{1}{r}\cdot\sin\theta\cdot\sin(\omega t -\beta r)
\end{empheq}

\subsubsection{Abgestrahlte Leistung im Fernfeld}
\begin{align*}
    P_\texttt{rad} = \frac{\eta {I_0}^2 \beta^2 (\Delta l')^2}{12\pi} = \frac{I_0^2\eta\pi}{3}\cdot \dfrac{\Delta l'^2}{\lambda^2}
\end{align*}

\subsection{Magnetischer Dipol}
\begin{align*}
    \Delta l & \rightarrow \beta\pi\ a^2
\end{align*}

\subsubsection{Fernfeld}
\begin{empheq}[box=\fbox]{align*}
    E & = \vec{\Phi}\cdot j\frac{\eta I_0\beta^2\pi a^2}{4\pi R}\cdot e^{-j\beta R}\cdot\sin\theta \\
    H & = \vec{\theta}\cdot j\frac{I_0\beta^2\pi a^2}{4\pi R}\cdot e^{-j\beta R}\cdot\sin\theta
\end{empheq}

\begin{align*}
    P_\texttt{rad} & = I_0^2\cdot\frac{\eta\pi}{3}\cdot\left(\frac{\beta\pi a^2}{\lambda}\right)^2                                      \\
                   & = I_0^2\cdot\frac{4}{3}\cdot Z_F\cdot\frac{\pi^5}{a^4\cdot\sqrt{\varepsilon_r^3}}\cdot\left(\frac{f}{c_0}\right)^4
\end{align*}

\subsubsection{Lineare Antenne}
\begin{align*}
    I(z') & = I_0\cdot\sin\left[\beta\left(\frac{L}{2}-|z'|\right)\right]
\end{align*}

\subsubsection{Dipolantenne}
\begin{align*}
    H   & = j\cdot\frac{I_0}{2\pi R}\cdot e^{-j\beta R}\cdot\frac{\cos\left[\left(\frac{\beta L}{2}\right)\cos\theta\right]-\cos\left(\frac{\beta L}{2}\right)}{\sin\theta} \\
    E   & = H\cdot\eta                                                                                                                                                      \\
    I_0 & = \sqrt{\frac{2\cdot P_{S}}{R_S}}
\end{align*}

\subsection{Bezugsantennen}
\[
    \boxed{g = 10 \cdot log(G) \si{dB}}
\]

mit $P_0$ : Eingangsleistung der Antenne

\begin{description}
    \item \textbf{\underline{G$\rightarrow$Bezugsantenne:}}

          Elementardipol  zu Kugelstrahler \[D = 1,50 \rightarrow g = 1,76\si{dBi}\]
          Halbwellendipol zu Kugelstrahler \[D = 1,64 \rightarrow g = 2,15\si{dBi}\]

    \item \textbf{\underline{EIRP}: Eqivalent \underline{Isoropic} Radiated Power}
          \[
              \text{EIRP} = P_0 \cdot G_i [\si{dBi}]
          \]

    \item \textbf{\underline{ERP}: Eqivalent Radiated Power (verlustloser Halbwellendipol)}
          \[
              \text{ERP} = P_0 \cdot G_d [\si{dBd}]
          \]
\end{description}

\subsection{Senden und Empfangen}
\begin{description}
    \item Senden = transmit = TX
    \item Empfangen = receive = RX
\end{description}

\begin{align*}
    \Aboxed{A_W & = G_{RX}\cdot\frac{\lambda^2}{4\pi}}                                     \\
    P_{RX}      & = S_{RX}\cdot A_W                                                        \\
                & = P_{TX}\cdot G_{TX}\cdot G_{RX}\cdot \left(\frac{\lambda}{4\pi r}\right)^2
\end{align*}

\subsubsection{Freiraumdämpfung/Freiraumdämpfungsmaß}
\begin{align*}
    F = \dfrac{P_{TX}}{P_{RX}} \cdot \left(\dfrac{4 \pi d}{\lambda}\right)^2 &\qquad [1] \\ 
    a_{0} = 20 \lg \left(\frac{4 \pi d}{\lambda}\right) =20 \lg \left(\frac{4 \pi d f}{c_{0}}\right)   &\qquad [\si{dB}]
\end{align*}

\subsubsection{Leistungspegel/Freiraumpegel}
\begin{align*}  
    L &= 10 \lg \frac{P}{1 mW} \qquad [\si{dBm}] \\
    L_{RX} &= L_{TX}+g_{TX}+g_{RX}-a_{0} \qquad [\si{dB}]
\end{align*}