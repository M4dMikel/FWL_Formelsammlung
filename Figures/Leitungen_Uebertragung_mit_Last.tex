\begin{center}
\resizebox{\columnwidth}{!}{
        \begin{circuitikz}%[american voltages]
            %Schaltbild
            \draw(0,0)
            to[V,v=$u_G(t)$](0,2)                               %Spannungsquelle
            to[R=$Z_g$](3,2)                                    %Quelleninnenwiderstand
            to[short,o-o](7,2)                                  %Leitung mit Knoten
            to[short](8 ,2)
            to[R=$Z_A$](8,0)                                    %Lastwiderstand
            to[short](7,0)                                      
            to[short,o-o](3,0)                                  %Leitung mit Knoten
            to[short](0,0);   

            %Knoten + Leitung Beschreibung
            \draw(3,2) node[above] {Eingang/Anfang};
            \draw(7,2) node[above] {Ausgang/Ende};
            \draw[decoration={brace},decorate]
                 (3,2.6) -- node[above=6pt] {$\underline{Z}_L$} (7,2.6);
            
            %linke gestrichelte linie
            \draw[dotted](3,0)--(3,-0.5) node[left]{$l=-d$};
            \draw[dotted](3,-0.5)--(3,-1) node[left]{$z=d$};
            \draw[dotted](3,-1)--(3,-1.5);

            %Pfeil in richtung l
            \draw[-latex](3,-0.5) -- (7,-0.5);
            \node at (4,-0.5)[above]{positiv $l$};
            
            %rechte gestrichelte Linue
            \draw[dotted](7,0)--(7,-0.5) node[right]{$l=0$};          
            \draw[dotted](7,-0.5)--(7,-1) node[right]{$z=0$};
            \draw[dotted](7,-1)--(7,-1.5);

            %pfeil in richtung z
            \draw[latex-](3,-1) -- (7,-1);
            \node at (6,-1)[above]{positiv $z$};

            %Pfeil in hinlaufende richtung
            \draw[-latex](3,1.25) -- (6.5,1.25);
            \node at (4.5,1.25)[above]{hinlaufende Welle};

            %Pfeil in rücklaufende richtung
            \draw[latex-](3.5,0.5) -- (7,0.5);
            \node at (5.5,0.5)[above]{rücklaufende Welle};
        \end{circuitikz}
}
\end{center}
