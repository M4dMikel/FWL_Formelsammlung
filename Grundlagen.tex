\section{Grundlagen}
\subsection{Differentialoperatoren}
\textbf{Nabla-Operator}
\begin{align*}
    \nabla & = \vec{\nabla} = 
    \begin{psmallmatrix}
        \partial  / \partial x \\
        \partial  / \partial y \\
        \partial  / \partial z
    \end{psmallmatrix}
\end{align*}

\textbf{Laplace-Operator}
\begin{align*}
    \varDelta  & = \vec{\nabla} \cdot \vec{\nabla} = \textrm{div (grad)} = 
    \dfrac{\partial ^2}{\partial x^2}+\dfrac{\partial ^2}{\partial y^2}+\dfrac{\partial ^2}{\partial z^2}
\end{align*}

\textbf{Divergenz} $\opdiv$: Vektorfeld $\rightarrow$ Skalar\\
\small{Quelldichte, gibt für jeden Punkt im Raum an, ob Feldlinien entstehen oder verschwinden.}
\begin{align*}
    \boxed{\opdiv \vec{F} = \nabla \cdot \vec{F}}   =  \dfrac{\partial F_x}{\partial x} 
    + \dfrac{\partial F_y}{\partial y} + \dfrac{\partial F_z}{\partial z}\\ 
                                 \begin{cases}
    > 0 \quad\Rightarrow \textnormal{Quelle}  \\
    < 0 \quad\Rightarrow \textnormal{Senke} \\
    = 0 \quad\Rightarrow \textnormal{quellenfrei} 
\end{cases}                                      
\end{align*}\\
% Bsp: $\opdiv \vec{B} = 0$, da mag. Felder in sich geschlossen.\\

\textbf{Rotation} $\oprot$: Vektorfeld $\rightarrow$ Vektorfeld\\ 
\small{Wirbeldichte, gibt für jeden Punkt im Raum Betrag und Richtung der Rotationsgeschwindigkeit an.}
\begin{align*}
\boxed{\oprot \vec{F} = \nabla \times \vec{F} } = 
\begin{pmatrix}
    \dfrac{\partial F_z}{\partial y} - \dfrac{\partial F_y}{\partial z} \\
    \dfrac{\partial F_x}{\partial z} - \dfrac{\partial F_z}{\partial x} \\
    \dfrac{\partial F_y}{\partial x} - \dfrac{\partial F_x}{\partial y}
\end{pmatrix} =
\begin{vmatrix}
    \vec{e}_x & \vec{e}_y & \vec{e}_z \\
    \dfrac{\partial}{\partial x} & \dfrac{\partial }{\partial x} & \dfrac{\partial }{\partial z} \\
    \vec{F}_x & \vec{F}_y & \vec{F}_z
\end{vmatrix}
\end{align*}\\
% Bsp: energieerhaltende (konservatives) Felder $ \rightarrow \oprot = 0$\\

\textbf{Gradient} $\opgrad$: Skalarfeld $\rightarrow$ Vektor/Gradientenfeld\\ 
\small{zeigt in Richtung steilster Anstieg von $\phi$}
\begin{align*}                                                                                          
    \boxed{\opgrad \phi = \nabla \cdot \phi }=  
    % \hspace{4.5ex}
    \begin{psmallmatrix}
        \partial \phi / \partial x \\
        \partial \phi / \partial y \\
        \partial \phi / \partial z
        % \dfrac{\partial G}{\partial x}\\
        % \dfrac{\partial G}{\partial y}\\
        % \dfrac{\partial G}{\partial z}
    \end{psmallmatrix}
    = \dfrac{\partial \phi}{\partial x} \vec{e}_x + \dfrac{\partial \phi}{\partial y} \vec{e}_y + 
    \dfrac{\partial \phi}{\partial z} \vec{e}_z  
\end{align*}

\subsubsection{Rechenregeln}
$\phi, \psi$: Skalarfelder \qquad $\vec{A}, \vec{B}$: Vektorfelder
\begin{align*}
     & \nabla \cdot (\vec{A} \times \vec{B}) & = & \qquad (\nabla \times \vec{A})\cdot\vec{B} - (\nabla\times\vec{B})\cdot\vec{A} \\
     & \nabla \cdot (\phi \cdot \psi)        & = & \qquad \phi (\nabla \psi) + \psi( \nabla \phi)                                  \\
     & \nabla \cdot (\phi \cdot \vec{A})           & = & \qquad \phi (\nabla \vec{A}) + \vec{A}(\nabla \phi)                             \\
     & \nabla \times (\phi \cdot \vec{A})          & = & \qquad \nabla \phi \times \vec{A} + \phi (\nabla \times \vec{A})                        
    %  & \oprot \opgrad f                      & = & \qquad 0 \Rightarrow\textnormal{Gradientenfeld Quellenfrei}                    \\
    %  & \opdiv \oprot \vec{f}                 & = & \qquad 0 \Rightarrow\textnormal{Wirbelfeld Quellenfrei}
\end{align*}

\subsubsection{Spezielle Vektorfelder}
quellenfreies Vektorfeld $\rightarrow$ Vektorpotential:
\begin{align*}
\opdiv \vec{F} = \opdiv (\oprot \vec{E}) = 0 \quad \Leftrightarrow \quad  \vec{F} = \oprot \vec{E}
\end{align*}
wirbelfreies Vektorfeld $\rightarrow$ Skalarpotential: 
\begin{align*}
    \oprot \vec{F} = \oprot (\opgrad \phi) = 0 \quad \Leftrightarrow \quad  \vec{F} = \opgrad \phi
\end{align*}
quellen- und wirbelfreies Vektorfeld:
\begin{align*}
    \oprot \vec{F}  & = 0 \quad \opdiv \vec{F} = 0\\
    \opdiv (\opgrad \phi) & = \varDelta \phi = 0 \quad \Leftrightarrow \quad  \vec{F} = \opgrad \phi
\end{align*}

% Feldänderung bei Bewegung
% \begin{align*}
%     \Delta G & = \dfrac{\partial G}{\partial x} \Delta x + \dfrac{\partial G}{\partial y} \Delta y + \dfrac{\partial G}{\partial z} \Delta z \\
%              & = dG = \opgrad G \cdot d \vec{s}
% \end{align*}

% \columnbreak

\subsection{Vektorrechnung}
\subsubsection{Betrag, Richtungswinkel, Normierung}
\textbf{Betrag}
\begin{align*}
    \vert \vec{r}  \vert & = r = \sqrt{r^2_x + r^2_y + r^2_z}
\end{align*}
\textbf{Richtungswinkel}
\begin{align*}
    \cos(\alpha) = \dfrac{a_x}{\vert \vec{a} \vert} \qquad \cos(\beta) = \dfrac{a_y}{\vert \vec{a} \vert} \qquad
        \cos(\gamma) = \dfrac{a_z}{\vert \vec{a} \vert}
    \end{align*}
    \textbf{Normierung, Einheitsvektor}
    \begin{align*}
        \vec{e}_a =  \dfrac{\vec{a}}{\vert \vec{a} \vert}, \quad \vert \vec{e}_a \vert = 1
    \end{align*}
    
    \subsubsection{Skalarprodukt}
    \begin{align*}
        \vec{a} \cdot \vec{b} & = |\vec{a}| \cdot |\vec{b}| \cdot cos(\varphi) \qquad \vec{a} \cdot \vec{b}  = 0\\
        cos(\varphi)          &  = \dfrac{\vec{a} \cdot \vec{b}}{|\vec{a}| \cdot |\vec{b}|} = \dfrac{a_x \cdot b_x + a_y \cdot b_y + a_z \cdot b_z}{|\vec{a}| \cdot |\vec{b}|}
    \end{align*}

    \subsubsection{Kreuzprodukt}
    
    \begingroup
        \renewcommand*{\arraystretch}{.95}
        \begin{align*}
        A_{Para} & = \vert \vec{c} \vert = \vert \vec{a} \times \vec{b} \vert = \vert \vec{a} \vert \cdot \vert \vec{b} \vert \cdot \sin(\varphi)\\
        \vec{a}\times\vec{b} & =
                \begin{pmatrix}
                    a_x \\
                    a_y \\
                    a_z
                \end{pmatrix}
                \times
                \begin{pmatrix}
                    b_x \\
                    b_y \\
                    b_z
                \end{pmatrix} =
                \begin{pmatrix}
                    a_yb_z-a_zb_y \\
                    a_zb_x-a_xb_z \\
                    a_xb_y-a_yb_x
                \end{pmatrix}
            \end{align*}
    \endgroup
    
    \subsection{Logarithmische Maße}
    Feldgröße $U_n$: Spannung, Strom, Schalldruck \\
    Leistungsgröße $P_n$: Energie, Intensität, Leistung

    % {\samepage
    % \begin{itemize}
    %     \setlength\itemsep{0pt}
    %     \item $\si{dBm} \hat=  1\si{mW}$
    %         \item $\si{dB\mu} V \hat= 1\si{\mu V}$
    %         \item $\si{dBmV} \hat{=} 1mV$
    %         \item $\si{dBi} \rightarrow$ Isotropic
    % \end{itemize}
        % }
        \begin{description} 
            \item \textbf{Dezibel} [dB]
            \begin{flalign*}
                1 \, \si{dB} & =  0,1151 \, \si{Np} \\  
                x & = 20 \cdot \log_{} \dfrac{U_1}{U_2} \,[\si{dB}] & x & = 10 \cdot \log_{}  \dfrac{P_1}{P_2} \,[\si{dB}] & \\
                U_1 & = U_2 \cdot 10^{\frac{x}{20\si{dB}}} & P_1   & = P_2 \cdot  10^{\frac{x}{10\si{dB}}} &               
            \end{flalign*}
            
            \item \textbf{Neper} [Np]
            \begin{flalign*}
                1 \si{Np} & =  8,686 \si{dB} \\   
                x & = \ln \dfrac{U_1}{U_2} \,[\si{Np}] & x & = \dfrac{1}{2} \cdot \ln \dfrac{P_1}{P_2} \,[\si{Np}] & \\
                U_1       & = U_2 \cdot e^{x}                   & P_1   & = P_2 \cdot  e^{2x} &
                % =\dfrac{20}{\ln 10} \cdot \ln \left( \dfrac{U_1}{U_2}\right)
            \end{flalign*}

            \item \textbf{Umrechnung}
            
            \begin{tabular}{l|l|l}
                Faktor & Energiegröße $P_n$ & Feldgröße $U_n$\\
                \hline
                1
            \end{tabular}
                
        \end{description}



% \subsection{Begriffe}
% \begin{tabularx}{0.45\textwidth}{>{\hsize=.1\hsize}X|>{\hsize=.5\hsize}X|>{\hsize=.4\hsize}X}
%            & Begriff           & Beschreibung \\
%     \hline
%     $\rho$ & Raumladungsdichte &              \\
% \end{tabularx}




\subsection{Vergleich/Umrechnung}
\begin{tabularx}{0.45\textwidth}{>{\hsize=.46\hsize}X|>{\hsize=.27\hsize}X|>{\hsize=.27\hsize}X}
    Kart.                                                                                & Zyl.             & Kug.                            \\
    \specialrule{1.5pt}{0pt}{0pt}
    $x$                                                                                  & $r \cos \varphi$ & $r \sin \vartheta \cos \varphi$ \\
    \hline
    $y$                                                                                  & $r \sin \varphi$ & $r \sin \vartheta \sin \varphi$ \\
    \hline
    $z$                                                                                  & $z$              & $r \cos \vartheta$              \\
    \specialrule{1.5pt}{0pt}{0pt}
    $\sqrt{x^{2}+y^{2}}$                                                                 & $r$              &                                 \\
    \hline
    $\arctan \frac{y}{x}$                                                                & $\varphi$        &                                 \\
    \hline
    $z$                                                                                  & $z$              &                                 \\
    \hline
    $d x \cos \varphi+d y \sin \varphi$                                                  & $dr$             &                                 \\
    \hline
    $d y \cos \varphi-d x \sin \varphi$                                                  & $r d\varphi$     &                                 \\
    \hline
    $dz$                                                                                 & $dz$             &                                 \\
    \specialrule{1.5pt}{0pt}{0pt}
    $\sqrt{x^{2}+y^{2}+z^{2}}$                                                           &                  & $r$                             \\
    \hline
    $\arctan \frac{y}{x}$                                                                &                  & $\varphi$                       \\
    \hline
    $\arctan \frac{\sqrt{x^{2}+y^{2}}}{z}$                                               &                  & $\vartheta$                     \\
    \hline
    $d x \sin \vartheta \cos \varphi+d y \sin \vartheta \sin \varphi+d z \cos \vartheta$ &                  & $dr$                            \\
    \hline
    $d y \cos \varphi-d x \sin \varphi$                                                  &                  & $r \sin \vartheta d \varphi$    \\
    \hline
    $d x \cos \vartheta \cos \varphi+d y \cos \vartheta \sin \varphi-d z \sin \vartheta$ &                  & $r d \vartheta$                 \\
\end{tabularx}

\subsection{Randbedingung}
\begin{tabularx}{0.45\textwidth}{>{\hsize=.3\hsize}X|>{\hsize=.7\hsize}X}
    Dirichlet-RB & Funktion nimmt an den Rändern einen bestimmten Wert an (Bsp.: $\rho_r = 5V$) \\
    \hline
    Neumann-RB   & Die Normalableitung der Fkt. nimmt an den Rändern einen bestimmten Wert an   \\
\end{tabularx}