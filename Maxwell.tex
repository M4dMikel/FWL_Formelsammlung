\section{Maxwell-Gleichungen}

\includegraphics[width=\columnwidth]{Figures/Integralsatz.png}

\textbf{Amperesches- /Durchflutungsgesetz:}

\begin{tabularx}{\textwidth}{>{\hsize=.5\hsize}X>{\hsize=.5\hsize}X}
    Elek. Strom ist Ursache für ein magn. Wirbelfeld. & $\boxed{\oint_s \vec{H} \cdot d \vec{s} = \Theta = I = \iint_A \vec{J} \cdot d \vec{A} = \frac{d\Phi_e}{dt}}$ \\
\end{tabularx}

\textbf{Induktionsgesetz:}

\begin{tabularx}{\textwidth}{>{\hsize=.5\hsize}X>{\hsize=.5\hsize}X}    
    Ein sich zeitlich änderndes Magnetfeld erzeugt ein elek. Wirbelfeld. & $\boxed{\oint_s{\vec{E} \cdot d\vec{s}} = u_{ind} = -\frac{d}{dt}\iint_A{\vec{B} \cdot d\vec{A}} = -\frac{d\Phi_m}{dt}}$                          \\
                                                                               & $\boxed{rot{\vec{E}} = -\frac{\partial\vec{B}}{\partial t} = -\mu\cdot\frac{\partial\vec{H}}{\partial t} = -j\omega\mu\vec{H}}$
\end{tabularx}

\textbf{Differentielles ohmsches Gesetz:}

\begin{tabularx}{\textwidth}{>{\hsize=.5\hsize}X>{\hsize=.5\hsize}X}
    Bewegte elektrische Ladung erzeugt Magnetfeld & $\boxed{ rot \vec{H} = \vec{J} = \kappa \cdot \vec{E}} $
\end{tabularx}

Bei isotropen Stoffen sind $\varepsilon$ u. $\mu$ Skalare:
\[
    \varepsilon = \varepsilon_0 \cdot \varepsilon_r \qquad \mu = \mu_0 \cdot \mu_r
\]

Zeitbereich: $ \dfrac{\partial}{\partial t} $ \qquad \qquad 
Harmonischer Frequenzbereich (komplexe Berechnung): $ jw $

\subsection{Integralsätze}
\begin{description}
    \setlength{\itemsep}{1pt}
    \item Fundamentalsatz der Analysis
    \item Gauß: Vektorfeld das aus Oberfläche von Volumen strömt muss aus Quelle in Volumen
    \item Stokes: innere Wirbel kompensieren sich $\rightarrow$ nur den Rand betrachten.
\end{description}
\begin{align*}
    \int_{a}^b \opgrad F \cdot d \vec{s}     & = F(b) - F(a)                                  \\
    \iiint_V \opdiv \vec{A} \cdot dV         & = \oiint_{ \partial V} \vec{A} \cdot d \vec{a} \\
    \iint_{A} \oprot \vec{A} \cdot d \vec{a} & = \oint_{ \partial A} \vec{A} \cdot d \vec{r}
\end{align*}
