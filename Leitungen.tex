\section{Leitungen}
\textbf{\color{red}{Leitungen gehen HIN und ZURÜCK!!!}\\
    \color{red}{Länge verdoppeln!!!}
}
\subsection{Leitungsparameter}

{\small\[
        \sigma = \text{Leitwert des Dielektr.} \qquad \sigma_c = \text{Leitwert des Leiters}
    \]}

\subsubsection{Doppelleitung:}
{\small\[
        a = \text{Leiter Radius} \qquad d = \text{Abstand zw. den Leitern} \\
    \]}

\input{Figures/Doppelleitung.tex}
{\renewcommand*{\arraystretch}{0.2}
    \begin{tabularx}{0.5\columnwidth}{|X|}
        \hline
        \[R  = \frac{1}{\pi a\delta\sigma_c}\]              \\
        \hline
        \[L = \frac{\mu}{\pi} \cosh^{-1}\frac{d}{2a}\]      \\
        \hline
        \[G = \frac{\pi\sigma}{\cosh^{-1}(^d/_{2a})}\]      \\
        \hline
        \[C = \frac{\pi\varepsilon}{\cosh^{-1}(^d/_{2a})}\] \\
        \hline
    \end{tabularx}}

\subsubsection{Koaxial Leitung}
{\small\[
        a = \text{innen Radius} \qquad b = \text{außen Radius} \\
    \]}

\begin{align*}
    \vec{H}(r, z) & = \frac{I}{2\pi r}\cdot e^{-j\beta z}\cdot\vec{e}_\varphi                    \\
    \vec{E}(r, z) & = \frac{I}{2\pi r}\cdot Z_F\cdot e^{-j\beta z} \cdot\vec{e}_r                \\
                  & = \frac{\hat{U}}{r \cdot\ln{(^{2b}/_{2a})}}\cdot e^{-j\beta z}\cdot\vec{e}_r
\end{align*}
\input{Figures/Koaxialleitung.tex}
{\renewcommand*{\arraystretch}{0.2}
    \begin{tabularx}{0.5\columnwidth}{|X|}
        \hline
        \[R=\frac{1}{2\pi\delta\sigma_c}\left[\frac{1}{a}+\frac{1}{b}\right]\] \\
        \hline
        \[L=\frac{\mu}{2\pi}\ln\frac{b}{a}\]                                   \\
        \hline
        \[G=\frac{2\pi\sigma}{\ln(^b/_a)}\]                                    \\
        \hline
        \[C=\frac{2\pi\varepsilon}{\ln(^b/_a)}\]                               \\
        \hline
    \end{tabularx}}
    \linebreak
    \linebreak
    \linebreak
    \linebreak
    \linebreak
\subsubsection{Parallele Platten}
{\small\[
        w  = \text{Platten Breite} \qquad d  = \text{Abstand zw. Platten}
    \]}

\begin{align*}
                  & \text{Für Sinus-Anregung:}                                                                                                 \\
    I             & = \frac{U}{Z_L} = \underbrace{\frac{U_0}{Z_L}}_{I_0}\cdot e^{-j\beta z\cdot e^{j\omega t}}                         \\
    U             & = \int \vec{E} d\vec{s} \stackrel{w\gg d}{=} E\cdot d \rightarrow E = \frac{U_0}{d}\cdot^{-j\beta z}\cdot\vec{e}_x \\
    I             & = \oint \vec{H} d\vec{s} =  H\cdot w \rightarrow H = \frac{I_0}{w}\cdot^{-j\beta z}\cdot\vec{e}_y                  \\
    \vec{E}(r, z) & = \frac{I}{2\pi r}\cdot Z_F\cdot e^{-j\beta z} \cdot\vec{e}_r                                                      \\
                  & = \frac{\hat{U}}{r \cdot\ln{(^{2b}/_{2a})}}\cdot e^{-j\beta z}\cdot\vec{e}_r
\end{align*}

\input{Figures/Parallele_Platten.tex}
{\renewcommand*{\arraystretch}{0.2}
    \begin{tabularx}{0.5\columnwidth}{|X|}
        \hline
        \[R=\frac{2}{w\delta\sigma}\] \\
        \hline
        \[L=\frac{\mu d}{w}\]         \\
        \hline
        \[G=\frac{\sigma w}{d}\]      \\
        \hline
        \[C=\frac{w\varepsilon}{d}\]  \\
        \hline
    \end{tabularx}
}

\vspace{1ex}
Für beliebige Leitergeometrie gelten folgende Zusammenhänge:
\[
    LC = \mu\varepsilon \quad \text{und} \quad \frac{G}{C} = \frac{\sigma}{\varepsilon}
\]

\subsection{Allgemeine Lösung Leitungsgleichung}
\begin{align*}
    \underline{U}(z)   & = U_h e^{\underline{\gamma} z} + U_r e^{-\underline{\gamma} z} = U_h e^{\underline{\gamma} d} + U_r e^{-\underline{\gamma} d}                       \\
    \underline{I}(z)   & = I_h e^{\underline{\gamma} z} + I_r e^{-\underline{\gamma} z} = \frac{U_h}{Z_L}e^{\underline{\gamma} d} - \frac{U_r}{Z_L}e^{-\underline{\gamma} d} \\
    \underline{Z}_L    & = \frac{U_h}{I_h} = \sqrt{ \frac{R + j \omega L}{G + j \omega C}}                                                                                   \\
    \underline{\gamma} & = j \omega \sqrt{LC} \cdot \sqrt{ \frac{RG}{j^2 \omega^2 LC} + \frac{G}{j \omega C} + \frac{R}{j \omega L} + 1}                                     \\
    \lambda            & = \frac{2 \pi}{\beta}                                                                                                                               \\
    v_p                & = \frac{\omega}{\beta}                                                                                                                              \\
    l_\texttt{elektr.} & = \beta \cdot l                                                                                                                                     \\
    \alpha             & = \omega \cdot \sqrt{\dfrac{\mu \varepsilon}{2}\cdot \left(\sqrt{1+\dfrac{\sigma^2}{\omega^2\cdot\varepsilon^2}}{\color{red}{-}}1\right)}           \\
    \beta              & = \omega \cdot \sqrt{\dfrac{\mu \varepsilon}{2}\cdot \left(\sqrt{1+\dfrac{\sigma^2}{\omega^2\cdot\varepsilon^2}}{\color{green}{+}}1\right)}
\end{align*}

\subsubsection{Verlustlose Übertragungsleitung}
\begin{align*}
    \underline{\gamma} & = j\omega\sqrt{LC}= j\beta                                                                                           \\
    Z_L                & =\frac{U_h}{U_r}       = \sqrt{\frac{L}{C}}                                                                          \\
    v_p                & = \frac{\omega}{\beta} = \frac{1}{\sqrt{LC}}= \frac{1}{\sqrt{\mu\varepsilon}}= \frac{c_0}{\sqrt{\mu_r\varepsilon_r}} \\
    \lambda            & = \frac{2\pi}{\beta}=\frac{1}{f\sqrt{LC}}= \frac{v_p}{f}= \frac{c_0}{f\sqrt{\mu_r\varepsilon_r}}
\end{align*}

\subsubsection{vernachlässigbarer Widerstandsbelag}
\includegraphics[width=\columnwidth]{Figures/vernachlaessigbarerWiderstandsbelag.png}


\subsubsection{vernachlässigbarer Leitwertbelag}
\includegraphics[width=\columnwidth]{Figures/vernachlaessigbarerLeiterwertbelag.png}

\subsection{Übertragungsleitung mit Last}

\input{Figures/Uebertragung_mit_Last.tex}

\begin{align*}
    U(z) & = U_h\cdot  e^{\underline{\gamma} z} + U_r\cdot  e^{-\underline{\gamma} z} = U_h\cdot  e^{\underline{\gamma} d} + U_r\cdot  e^{-\underline{\gamma} d}           \\
    I(z) & = I_h\cdot  e^{\underline{\gamma} z} + I_r\cdot  e^{-\underline{\gamma} z} = \frac{U_h}{Z_L}e^{\underline{\gamma} d} - \frac{U_r}{Z_L}e^{-\underline{\gamma} d}
\end{align*}

\subsubsection{Vorgehen Eingangswiderstand}
Wenn mit Smithdiagramm gearbeitet wird liefert dieses Schritte \ref{Ref L_anfang} und \ref{Bestimmen Z_E}
\begin{enumerate}
    \item Lastimpedanz
          \[ \underline{Z}_A = \dfrac{1}{\frac{1}{R_A} + j \omega C_A} \]
    \item Reflexion am Leitungsende
          \[ \underline{r}_A = \underline{r}(z=0) = \dfrac{Z_A - \underline{Z}_L}{Z_A + \underline{Z}_L} \]
    \item Reflexion am Leitungsanfang \label{Ref L_anfang}
          \[ \underline{r}_E = \underline{r}(z=d) =  \underline{r}_A \cdot e^{-j 2 \beta d}\]
    \item Bestimmung der Impedanz \label{Bestimmen Z_E}
          \[ \underline{Z}_E = \underline{Z}_L \cdot \dfrac{1 + \underline{r}_E}{1 - \underline{r}_E}\]
    \item Eingangswiderstand
          \[ \underline{Z}_E = \dfrac{1}{\frac{1}{\underline{Z}_E} + j \omega C_E}\]
\end{enumerate}


\subsubsection{Reflexionsfaktor entlang einer Leitung}
\begin{align*}
    r_E    & = r_A  ^{-2\underline{\gamma} l} = r_A  e^{-2\alpha l} e^{-j2\beta l}                                                     \\
    \alpha & = -\frac{\ln(r_A)}{2l} [\si{Np/m}]                                    & \beta & = \dfrac{\phi_2 -\phi_1}{2l} [\si{rad/m}]
\end{align*}

\subsubsection{Stehwellenverhältnis}
\begin{align*}
    \mathrm{SWR} = \frac{U_\text{max}}{U_\text{min}} =
    \frac{I_\text{max}}{I_\text{min}} = \frac{1+|r(z)|}{1-|r(z)|} =
    \frac{|U_H|+|U_R|}{|U_H|-|U_R|}
\end{align*}

\subsubsection{Position von Extrema}
\begin{gather*}
    \boxed{r_A = |r_A|\cdot e^{-j\theta_r}}\rightarrow\theta_r\text{ in rad}\\
    f_\texttt{min}\rightarrow \text{Minimum(Knoten) der Spannungen}\\
    f_\texttt{max}\rightarrow \text{Maximum(Bäuche) der Spannungen}
\end{gather*}
\begin{align*}
    \lambda_\texttt{min/max} & = \frac{c_0}{f_\texttt{min/max}\sqrt{\mu_{r1}\varepsilon_{r1}}}                                                                                                 \\
    z_\texttt{min}           & =\frac{-n\cdot\lambda_\texttt{min}}{2}                                        \qquad\rightarrow n = -\frac{2z}{\lambda_\texttt{min}}                            \\
    z_\texttt{max}           & =\frac{-(2n+1)\lambda_\texttt{max}}{4}                                        \qquad\rightarrow n = -\frac{4z+\lambda_\texttt{max}}{2\cdot\lambda_\texttt{max}} \\
    z                        & = \frac{\lambda_\texttt{min}\cdot\lambda_\texttt{max}}{4(\lambda_\texttt{min}-\lambda_\texttt{max})}
\end{align*}

\vspace{20pt}
\textbf{Spezialfälle auf der nächsten Seiten}
\pagebreak

\subsubsection{Spezialfall: Angepasste Leitung}
\begin{align*}
    Z_A          & = Z_L = Z(z)                              \\
    r_A          & = 0\qquad\rightarrow\text{reflexionsfrei} \\
    \mathrm{SWR} & = 1                                       \\
    U(z)         & = U_h\cdot e ^{j\beta z}                  \\
    I(z)         & = I_h \cdot e^{j\beta z}                  \\
                 & = \frac{U_h}{Z_L}\cdot e^{j\beta z}
\end{align*}

\subsubsection{Spezialfall: Kurzgeschlossene Leitung}
\begin{align*}
    Z_A          & = 0                                                                        \\
    Z(z)         & = j Z_L\cdot\tan(\beta z)        \qquad\rightarrow\text{rein imaginär}     \\
    r_A          & = -1                                                                       \\
    \mathrm{SWR} & = \infty                                                                   \\
    U(z)         & = U_h\cdot 2j\sin(\beta z)    \qquad\rightarrow U(z=0)=0                   \\
    I(z)         & = U_h\cdot 2\cos(\beta z)    \qquad\rightarrow I(z=0)=I_A=\frac{2U_h}{Z_L}
\end{align*}

\subsubsection{Spezialfall: Leerlaufende Leitung}
\begin{align*}
    Z_A          & = \infty                                                                         \\
    Z(z)         & = -jZ_L\cdot \cot(\beta z) \qquad\rightarrow\text{rein imaginär}                 \\
    r_A          & = 1                                                                              \\
    \mathrm{SWR} & = \infty                                                                         \\
    U(z)         & = U_h\cdot 2\cos(\beta z) \qquad\rightarrow U(z=0)=0                             \\
    I(z)         & = U_h\cdot 2j\sin(\beta z) \qquad\rightarrow I(z=0)=I_A = \frac{2\cdot U_h}{Z_L}
\end{align*}

\subsubsection{Spezialfall: Ohm'sch abgeschlossene Leitung}
\[
    r_A = \texttt{reell}
\]

\begin{align*}
    \underline{R_A > Z_L} & \rightarrow\theta_r = 0 \rightarrow r_A \texttt{ ist negativ} \\
                          & \rightarrow z_\texttt{max}=\frac{\lambda}{2}\cdot n
\end{align*}

\begin{align*}
    \underline{R_A < Z_L} & \rightarrow\theta_r = \pi                           \\
                          & \rightarrow z_\texttt{min}=\frac{\lambda}{2}\cdot n
\end{align*}

\columnbreak
\subsection{Mehrfachreflexionen bei fehlender Anpassung}
\input{Figures/Mehrfach_Refexion.tex}
\begin{align*}
    %u_{1h} & = u_G\cdot\frac{ Z_L}{R_I + Z_L}            \\
    u_{1r} & = r_A\cdot u_{1h}                                \\
    u_{2h} & = r_I\cdot u_{1r} = r_I\cdot r_A\cdot u_{1h}     \\
    u_{2r} & = r_A\cdot u_{2h} = r_I\cdot r_A^2\cdot u_{1h}   \\
    u_{3h} & = r_I\cdot u_{2r} = r_I^2\cdot r_A^2\cdot u_{1h}
\end{align*}
\input{Figures/Mehrfach_Reflexion_Circuit.tex}
\begin{align*}
     & \text{Reflexionsfaktor Leitungsanfang: } & \underline{r}_I & = \frac{R_I - Z_L}{R_I + Z_L}                 \\
     & \text{Reflexionsfaktor Leitungsende: }   & \underline{r}_A & = \frac{R_A - Z_L}{R_A + Z_L}                 \\
     & \text{Signallaufzeit: }                  & t_d             & = \frac{l}{c_0}\cdot\sqrt{\mu_r\varepsilon_r} \\
     & \text{Hinlaufende Welle}                 & u_{1h}          & = \hat{u}_G \cdot\frac{Z_L}{Z_L+R_I}
\end{align*}
